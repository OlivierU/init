%!TEX program = <lualatex>
\documentclass[12pt]{article}


\usepackage[utf8]{inputenc}
\usepackage[english]{babel}
\usepackage{csquotes}
\usepackage{graphicx}
\usepackage{indentfirst}
\usepackage{fontspec}
\usepackage{fancyhdr}
\usepackage{titling}
\usepackage{lipsum}

\usepackage[
backend=biber,
style=apa,
]{biblatex}

\usepackage[style=iso]{datetime2}

\usepackage{geometry}
\geometry{a4paper, margin=1in}

\addbibresource{INIT.bib}

\setlength{\parindent}{0.5in}
\renewcommand{\baselinestretch}{2}
\setlength{\headheight}{15pt}

\title{%
	INIT \\
	\large Individually Neurostimulated Inhibition Training for AUD}

\author{Andrea Häfliger, Olivier Ulrich, Konstantinos Zervas}

\pagestyle{fancy}
\fancyhead{}
\fancyhead[R]{INIT}
\fancyfoot{}
\fancyfoot[R]{\thepage}
\fancyfoot[L]{\theauthor}

\begin{document}

\pagestyle{fancy}
\thispagestyle{empty}

\maketitle
\newpage
\tableofcontents
\newpage

\section{Research Plan: Summary}

The proposed study will evaluate the potential of network-targeted brain stimulation to increase the efficacy of inhibition training protocols in treating alcohol use disorder. By comparing inhibition training accompanied by individually targeted transcranial magnetic stimulation with both sham, conventional TMS, a standalone inhibition training protocol, and a control group receiving the standard of care in AUD treatment, we aim to assess both the feasability of such a regimen, as well as it's effect on treatment outcomes relating to AUD. The inhibition training uses a modified Go/NoGo task to improve patients' ability to inhibit pre-potent responses toward alcoholic stimuli and has been shown to reduce alcohol intake \parencite{houbenBeerNogoLearning2012}. Performance differences between AUD patients and healthy controls in regular Go/NoGo tasks using alcohol related stimuli has been shown to depend on a network of neuronal circuitry commonly thought of being involved in inhibitory processes \parencite{czaplaAlcoholdependentPatientsShow2017,volkowAddictionScienceUncovering2014,simmondsMetaanalysisGoNogo2008,luijtenSystematicReviewERP2014}. Network-targeted TMS has been suggested to be able to modulate neuronal activity in deeper lying brain regions, overcoming a critical hurdle of non-invasive brain stimulation \parencite{momiCognitiveEnhancementNetworkTargeted2020}. By using critical regions in inhibitory processes to build a functional connectivity map of inhibition we can identify individual cortical regions in each participant that can serve as gateways for TMS to target the underlying inhibitory network and thus increase inhibitory performance. This allows us to evaluate whether this facilitated inhibitory control can improve the effects of inhibition training for AUD.

\section{Research Plan}


\subsection{State of research}


\subsection{Detailed research plan}


\subsubsection{Objectives and hypotheses}

This project's objectives are essentialy twofold: for one, we intend to further the understanding of the neuronal circuitry involved in inhibitory processes related to alcohol specific stimuli by investigating functional connections between various subcortical regions known to be implicated in the act of inhibitory control, as well as search for further connections linking these areas to cortical regions that may be used as targets for non-invasive brain stimulation. Secondly, we intend to use these findings to identify patient specific cortical gateway regions to stimulate during established inhibition training session in order to investigate the stimulation's potential to bolster the improvement in inhibitory control performance evoked by these trainings.

The functional and connectional insights attained through the pre- and post-measures bear the potential of both broadening the spectrum of potentially relevant inhibitory circuitry, and deepening our understanding of the specific involvements and interactions between already identified areas in inhibition processes. Exemplarised through the interaction model of addiction and inhibition proposed by \textcite{volkowAddictionScienceUncovering2014}, this may result in a proposed inclusion of further regions, hitherto not having been directly implicated in these processes, or additional connections between already included regions. Due to the intervention based approach and the thorough nature of the proposed neurophysiological measurements in this study, a quantifying of certain functional connections and their implications in inhibitory performance specifically and in addiction more generally is a possibility as well.

The aforementioned intervention based approach forms the framework to realise the second of the proffered project's goals; evaluating the potential of transcranial magnetic stimulation in increasing the effects of inhibition training in patients suffering from AUD. Existing research in this field has investigated both the effect of non-invasive brain stimulation on addiction in general \parencite{antonelliTranscranialMagneticStimulation2021,mostafaviNoninvasiveBrainStimulation2020,zhangEffectsRepetitiveTranscranial2019}, and it's use in identifying and describing neuronal circuitry involved in inhibition \parencite{naim-feilCorticalInhibitionMotor2016,quoilinNeuralBasesInhibitory2021}. To date, however, there has not been any systematic inquiry into the effectiveness of TMS or other non-invasive brain stimulation in conjunction with behavioural inhibition training paradigms. Our study has set it's sight on this specific niche and aims to go one step further by also including network targeting methods to determine optimal TMS application and it's effect on inhibition training. These methods have been shown to allow targeting of specific networks usually unreachable by TMS by leveraging the functional connectivity underlying the circuitry in question and has been shown to be a fruitful method to improve cognitive performance in tasks related to the targeted networks \parencite{momiCognitiveEnhancementNetworkTargeted2020}. Based on the framework of \textcite{volkowAddictionScienceUncovering2014}, we thus posit that such network targeted TMS may be well suitable to target the neural circuitry involved in inhibition and thus has the potential of being an effective addition to the current, purely behavioural tasks used in inhibition training programs.

\subsubsection{Participants}

Meta analyses failed to show significant effect sizes when performing TMS on participants with AUD. For this reason, we first chose a small effectsize f of 0.1 with a minimum power of 0.8 and the default-setting for the expected correlation among the repeated measures of 0.5 for calculating the required sample size with G*Power. The resulting sample size reads 900 total, which did not seem adequat. We decided to adjust to a small to medium effect size of 0.2, which results in a sample size of 230 participants, 46 per condition. 
For this study, 230 left-handed men with adequate knowledge of the German language will be recruited. All exclusion criteria are listed below. 
•	No tattoos in the head area 
•	No MR-compliant iron/metals in/on the body
•	No dental implants (retainers are allowed)
•	No neurological or psychiatric history or current medical conditions, e.g. epilepsy
•	No Intake of any medication that is likely to lower seizure threshold
•	No excessive drug use besides alcohol
•	No diabetes, heart diseases or current infections
•	No sleep, memory or concentration disorders or sleep deprivation
•	No claustrophobia or sensitivity to noise
•	No dizziness or migraine 
•	No nearsightedness or farsightedness if not corrected by currently valid (recently prescribed) contact lenses
•	No participation in previous studies with the same content  
•	No vaccinations in the near future 

All participants are newly admitted patients diagnosed with AUD from two substance-abuse clinics near Bern: Klinik Südhang and Klinik Selhofen. To ensure that the diagnosis is accurate, all subjects complete the AUDIT questionnaire at the start of treatment. When they first check in the facility, they will be given an information sheet with explanations of the aims of this research. They will sign a declaration of consent for the researchers to collect and use the data needed for the present study. Furthermore, they are asked to fill out a questionnaire to ensure that the inclusion and exclusion criteria are met. Finally, participants must agree to be informed in case of incidental findings. Those who do not wish to do so will not be allowed to participate in the study.

\subsubsection{Study design, procedures, and measures}

To allow for a thorough later analysis of the trainings' effects, we will initially conduct a baseline measure of both inhibitory control as well as the individuals' reactions to alcoholic stimuli during an fMRI scan. Inihbitory control is measured by a modified Go/NoGo task using alcohol related stimuli as no-go signals. These kinds of Go/NoGo task find widespread use in the assesment of inhibitory control in a context of AUD and provide a robust measure to assess an individual's ability to inhibit an automatised response to alcohol related information \parencite{amesNeuralCorrelatesGo2014,bowleyEffectsInhibitoryControl2013,roseEffectsAlcoholInhibitory2008,simmondsMetaanalysisGoNogo2008}. Unlike the Go/NoGo task used in the actual inhibition training \parencite{houbenBeerNogoLearning2012}, this task uses the alcoholic stimuli, i.e. images depicting alcoholic beverages, as NoGo signals, which stand in contrast to pictures of non-alcoholic beverages that serve as Go signals. The participants are thus instructed to press a button whenever a Go signal appears, but not when a NoGo signal does. The stimuli are presented in a randomised order and a four to one ratio of Go and NoGo signals, respectively. This ratio has been shown to ensure the formation of a pre-potent response tendency towards Go signals \parencite{amesNeuralCorrelatesGo2014}. Since the task will be performed during an fMRI scan the timings of both the trials and the inter trial duration will be jittered.

Each participant's neurophysiological reaction to alcoholic stimuli will be measured using a simple cue reactivity task consisting of alcholic stimuli. 

These baseline measures will be conducted during an fMRI scan, following a T1 weighted structural scan to infer anatomical information of each participant. Based on the fMRI data collected during both the cue reactivity task and the Go/NoGo task, BOLD signals will be analysed to assess individual neurophysiological activity. For the Go/NoGo task, BOLD activity during sucessfully inhibited responses (correctly rejected NoGo trials) and correctly answered Go trials are compared, allowing the identification of regions involved in inhibitory control. Contrasts in the cue reactivity task will be formed between the fixation trials and the alcohol stimulus trials, thus providing information on the location and extent of the individual physiological response.

The functional data obtained during the Go/NoGo task will be used to conduct connectivity analyses between inhibitory regions and a series of candidate regions closer to the surface of the brain. These analyses are intended to identifiy potential gateway regions, which can be used as targets of non-invasive brain stimulation and would propagate the stimulation's effects to the lower laying inhibitory regions that cannot be targeted directly. The aim is to find the most positively correlated nodes of a functional connectivity map calculated from individual seed regions. The seeds will be chosen based on the BOLD activity during the inhibitory control task.

Potential gateway regions include the premotor cortex, the dorsolateral prefrontal cortex, and the orbitofrontal cortex. Based on these connectivity data for each participant, an individualised stimulation locus will be decided upon to be used during stimulation enhanced inhibition training.

\subsubsection{Data analysis}


\subsubsection{Ethical considerations}


\subsection{Timetable and milestones}


\subsection{Significance of planned research to the scientific community and to potential users}


\printbibliography  
\end{document}