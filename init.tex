%!TEX program = <lualatex>
\documentclass[12pt]{article}


\usepackage[utf8]{inputenc}
\usepackage[english]{babel}
\usepackage{csquotes}
\usepackage{graphicx}
\usepackage{indentfirst}
\usepackage{fontspec}
\usepackage{fancyhdr}
\usepackage{titling}
\usepackage{lipsum}

\usepackage[
backend=biber,
style=apa,
]{biblatex}

\usepackage[style=iso]{datetime2}

\usepackage{geometry}
\geometry{a4paper, margin=1in}

\addbibresource{INIT.bib}

\setlength{\parindent}{0.5in}
\renewcommand{\baselinestretch}{2}
\setlength{\headheight}{15pt}

\title{%
	INIT \\
	\large Individually Neurostimulated Inhibition Training for AUD}

\author{Andrea Häfliger, Olivier Ulrich, Konstantinos Zervas}

\pagestyle{fancy}
\fancyhead{}
\fancyhead[R]{INIT}
\fancyfoot{}
\fancyfoot[R]{\thepage}
\fancyfoot[L]{\theauthor}

\begin{document}

\pagestyle{fancy}
\thispagestyle{empty}

\maketitle
\newpage
\tableofcontents
\newpage

\section{Research Plan: Summary}

\lipsum[1-5]

\section{Research Plan}

\lipsum

\subsection{State of research}

\lipsum[1-3]

\subsection{Detailed research plan}

\lipsum[1-5]

\subsubsection{Objectives and hypotheses}

To further the understanding of the neuronal circuitry involved in inhibitory processes related to alcohol specific stimuli we intend to investigate functional connections between various subcortical regions known to be implicated in the act of inhibitory control, as well as search for further connections linking these areas to cortical regions that may be used as targets for non-invasive brain stimulation. We intend to use these findings to identify patient specific cortical gateway regions to stimulate during established inhibition training session in order to investigate the stimulation's potential to bolster the improvement in inhibitory control performance evoked by these trainings.

\subsubsection{Participants}

\lipsum[1-2]

\subsubsection{Study design, procedures, and measures}

To allow for a thorough later analysis of the trainings' effects, we will initially conduct a baseline measure of both inhibitory control as well as the individuals' reactions to alcoholic stimuli. Inihbitory control is measured by a modified Go/NoGo task using alcohol related stimuli as no-go signals. These kinds of Go/NoGo task find widespread use in the assesment of inhibitory control in a context of AUD and provide a robust measure to assess an individual's ability to inhibit an automatised response to alcohol related information. Each participant's neurophysiological reaction to alcoholic stimuli will be measured using a simple cue reactivity task consisting of alcholic stimuli. These baseline measures will be conducted during an fMRI scan, following a T1 weighted structural scan to infer anatomical information of each participant.
Based on the fMRI data collected during the cue reactivity task, BOLD signals will be analysed to assess individual neurophysiological activity in regions of interest during confrontation with AUD related stimuli.

\subsubsection{Data analysis}

\lipsum[1-3]

\subsubsection{Ethical considerations}

\lipsum[1-2]

\subsection{Timetable and milestones}

\lipsum[1-2]

\subsection{Significance of planned research to the scientific community and to potential users}

\lipsum[1-2]

\printbibliography  
\end{document}