%!TEX program = <lualatex>
\documentclass[12pt]{article}


\usepackage[utf8]{inputenc}
\usepackage[english]{babel}
\usepackage{csquotes}
\usepackage{graphicx}
\usepackage{indentfirst}
\usepackage{fontspec}
\usepackage{fancyhdr}
\usepackage{titling}
\usepackage{lipsum}

\usepackage[
backend=biber,
style=apa,
]{biblatex}

\usepackage[style=iso]{datetime2}

\usepackage{geometry}
\geometry{a4paper, margin=1in}

\addbibresource{INIT.bib}

\setlength{\parindent}{0.5in}
\renewcommand{\baselinestretch}{2}
\setlength{\headheight}{15pt}

\title{%
	INIT \\
	\large Individually Neurostimulated Inhibition Training for AUD}

\author{Andrea Häfliger, Olivier Ulrich, Konstantinos Zervas}

\pagestyle{fancy}
\fancyhead{}
\fancyhead[R]{INIT}
\fancyfoot{}
\fancyfoot[R]{\thepage}
\fancyfoot[L]{\theauthor}

\begin{document}

\pagestyle{fancy}
\thispagestyle{empty}

\maketitle
\newpage
\tableofcontents
\newpage

\section{Research Plan: Summary}

The proposed study will evaluate the potential of network-targeted brain stimulation to increase the efficacy of inhibition training protocols in treating alcohol use disorder. By comparing inhibition training accompanied by individually targeted transcranial magnetic stimulation with both sham, conventional TMS, a standalone inhibition training protocol, and a control group receiving the standard of care in AUD treatment, we aim to assess both the feasability of such a regimen, as well as it's effect on treatment outcomes relating to AUD. The inhibition training uses a modified Go/NoGo task to improve patients' ability to inhibit pre-potent responses toward alcoholic stimuli and has been shown to reduce alcohol intake \parencite{houbenBeerNogoLearning2012}. Performance differences between AUD patients and healthy controls in regular Go/NoGo tasks using alcohol related stimuli has been shown to depend on a network of neuronal circuitry commonly thought of being involved in inhibitory processes \parencite{czaplaAlcoholdependentPatientsShow2017,volkowAddictionScienceUncovering2014,simmondsMetaanalysisGoNogo2008,luijtenSystematicReviewERP2014}. Network-targeted TMS has been suggested to be able to modulate neuronal activity in deeper lying brain regions, overcoming a critical hurdle of non-invasive brain stimulation \parencite{momiCognitiveEnhancementNetworkTargeted2020}. By using critical regions in inhibitory processes to build a functional connectivity map of inhibition we can identify individual cortical regions in each participant that can serve as gateways for TMS to target the underlying inhibitory network and thus increase inhibitory performance. This allows us to evaluate whether this facilitated inhibitory control can improve the effects of inhibition training for AUD.

\section{Research Plan}


\subsection{State of research}


\subsection{Detailed research plan}


\subsubsection{Objectives and hypotheses}

This project's objectives are essentialy twofold: for one, we intend to further the understanding of the neuronal circuitry involved in inhibitory processes related to alcohol specific stimuli by investigating functional connections between various subcortical regions known to be implicated in the act of inhibitory control, as well as search for further connections linking these areas to cortical regions that may be used as targets for non-invasive brain stimulation. Secondly, we intend to use these findings to identify patient specific cortical gateway regions to stimulate during established inhibition training session in order to investigate the stimulation's potential to bolster the improvement in inhibitory control performance evoked by these trainings.

The functional and connectional insights attained through the pre- and post-measures bear the potential of both broadening the spectrum of potentially relevant inhibitory circuitry, and deepening our understanding of the specific involvements and interactions between already identified areas in inhibition processes. Exemplarised through the interaction model of addiction and inhibition proposed by \textcite{volkowAddictionScienceUncovering2014}, this may result in a proposed inclusion of further regions, hitherto not having been directly implicated in these processes, or additional connections between already included regions. Due to the intervention based approach and the thorough nature of the proposed neurophysiological measurements in this study, a quantifying of certain functional connections and their implications in inhibitory performance specifically and in addiction more generally is a possibility as well.

The aforementioned intervention based approach forms the framework to realise the second of the proffered project's goals; evaluating the potential of transcranial magnetic stimulation in increasing the effects of inhibition training in patients suffering from AUD. Existing research in this field has investigated both the effect of non-invasive brain stimulation on addiction in general \parencite{antonelliTranscranialMagneticStimulation2021,mostafaviNoninvasiveBrainStimulation2020,zhangEffectsRepetitiveTranscranial2019}, and it's use in identifying and describing neuronal circuitry involved in inhibition \parencite{naim-feilCorticalInhibitionMotor2016,quoilinNeuralBasesInhibitory2021}. To date, however, there has not been any systematic inquiry into the effectiveness of TMS or other non-invasive brain stimulation in conjunction with behavioural inhibition training paradigms. Our study has set it's sight on this specific niche and aims to go one step further by also including network targeting methods to determine optimal TMS application and it's effect on inhibition training. These methods have been shown to allow targeting of specific networks usually unreachable by TMS by leveraging the functional connectivity underlying the circuitry in question and has been shown to be a fruitful method to improve cognitive performance in tasks related to the targeted networks \parencite{momiCognitiveEnhancementNetworkTargeted2020}. Based on the framework of \textcite{volkowAddictionScienceUncovering2014}, we thus posit that such network targeted TMS may be well suitable to target the neural circuitry involved in inhibition and thus has the potential of being an effective addition to the current, purely behavioural tasks used in inhibition training programs.

\subsubsection{Participants}

Meta analyses failed to show significant effect sizes in studies that used TMS to treat patients with AUD \parencite{mostafaviNoninvasiveBrainStimulation2020, zhangEffectsRepetitiveTranscranial2019}. To calculate the required sample size using G*Power, we thus chose a small effect size of f=0.1 at first, which, along with a minimum power of 0.8 and the default setting for the expected correlation among the repeated measures of 0.5, resulted in a suggested sample size of 900 total. Since a sample size this large did not seem adequate, we adjusted the effect size to f=0.2, representing a small to medium effect. This resulted in a sample size of 230, or 46 participants per condition. 
To avoid any interferences from differences in lateralizations of language, only right-handed patients will be recruited. Exclusion criteria are defined as follows:

\begin{itemize}
\item No tattoos in the head area 
\item No MR-compliant iron/metals in/on the body
\item No dental implants (retainers are allowed)
\item No neurological or psychiatric history or current medical conditions, e.g. epilepsy
\item No Intake of any medication that is likely to lower seizure threshold
\item No excessive drug use besides alcohol
\item No diabetes, heart diseases or current infections
\item No sleep, memory or concentration disorders or sleep deprivation
\item No claustrophobia or sensitivity to noise
\item No dizziness or migraine 
\item No nearsightedness or farsightedness if not corrected by currently valid (recently prescribed) contact lenses
\item No participation in previous studies with the same content  
\item No vaccinations in the near future 
\end{itemize}

All participants will be recruited among newly admitted patients diagnosed with AUD in two substance-abuse clinics near Bern: Klinik Südhang and Klinik Selhofen. Upon first checking into the clinic, patients receive an information booklet explaining the aims of the study and sign a declaration of consent about the collection and use of data required in this project. They are also asked to fill out a questionnaire about the in- and exclusion criteria and about their agreement to be informed about incidental findings. Failure to do so will exclude them from participating.

All participants willing to take part in the study will be informed of the complete procedures that will take place including the MRI, rTMS, Go/No-Go task and the follow up questionnaire including the HDL and the TLFB.

\subsubsection{Study design, procedures, and measures}

To allow for a thorough later analysis of the trainings' effects, we will initially conduct a baseline measure of both inhibitory control as well as the individuals' reactions to alcoholic stimuli during an fMRI scan. Inihbitory control is measured by a modified Go/NoGo task using alcohol related stimuli as no-go signals. These kinds of Go/NoGo task find widespread use in the assesment of inhibitory control in a context of AUD and provide a robust measure to assess an individual's ability to inhibit an automatised response to alcohol related information \parencite{amesNeuralCorrelatesGo2014,bowleyEffectsInhibitoryControl2013,roseEffectsAlcoholInhibitory2008,simmondsMetaanalysisGoNogo2008}. Unlike the Go/NoGo task used in the actual inhibition training \parencite{houbenBeerNogoLearning2012}, this task uses the alcoholic stimuli, i.e. images depicting alcoholic beverages, as NoGo signals, which stand in contrast to pictures of non-alcoholic beverages that serve as Go signals. The participants are thus instructed to press a button whenever a Go signal appears, but not when a NoGo signal does. The stimuli are presented in a randomised order and a four to one ratio of Go and NoGo signals, respectively. This ratio has been shown to ensure the formation of a pre-potent response tendency towards Go signals \parencite{amesNeuralCorrelatesGo2014}. Since the task will be performed during an fMRI scan the timings of both the trials and the inter trial duration will be jittered.

Each participant's neurophysiological reaction to alcoholic stimuli will be measured using a simple cue reactivity task consisting of alcholic stimuli. 

These baseline measures will be conducted during an fMRI scan, following a T1 weighted structural scan to infer anatomical information of each participant. Based on the fMRI data collected during both the cue reactivity task and the Go/NoGo task, BOLD signals will be analysed to assess individual neurophysiological activity. For the Go/NoGo task, BOLD activity during sucessfully inhibited responses (correctly rejected NoGo trials) and correctly answered Go trials are compared, allowing the identification of regions involved in inhibitory control. Contrasts in the cue reactivity task will be formed between the fixation trials and the alcohol stimulus trials, thus providing information on the location and extent of the individual physiological response.

The functional data obtained during the Go/NoGo task will be used to conduct connectivity analyses between inhibitory regions and a series of candidate regions closer to the surface of the brain. These analyses are intended to identifiy potential gateway regions, which can be used as targets of non-invasive brain stimulation and would propagate the stimulation's effects to the lower laying inhibitory regions that cannot be targeted directly. The aim is to find the most positively correlated nodes of a functional connectivity map calculated from individual seed regions. The seeds will be chosen based on the BOLD activity during the inhibitory control task.

Potential gateway regions include the premotor cortex, the dorsolateral prefrontal cortex, and the orbitofrontal cortex. Based on these connectivity data for each participant, an individualised stimulation locus will be decided upon to be used during stimulation enhanced inhibition training.

\subsubsection{Data analysis}

Based on our hypotheses, we excpect to find effects both within the experimental groups, due to the general effectiveness of AUD interventions, and between them, due to the potential of increasing training effectiveness through TMS. To test for these effects, we can construct multiple ANOVAs based on our outcome measures using time as a within factor and the experimental groups as a between factor. This allows us to investigate any significant interactions between time and group for each measure in detail, using post-hoc group comarisons to asses the iterative effectiveness of each of the proposed interventions.

For each of the outcome measures, different within factors need to be added, for the Go/NoGo task and the cue reactivity taks the within factors will include the stimulus type and time, and for the structural data it will only include time.

\subsubsection{Ethical considerations}
All inclusion and exclusion criteria that ensure the participants are in good health condition are listed above.

As there is a shamTMS conditon in our study design to exclude all the placebo effects that trace to the fact that something is done with the participants, participants in the shamTMS condition will be informed as if they were in the experimental condition. This is the only sort of deception included in this study. 

The MRI sessions don't pose any health risks. The narrowness of the scanner tube and it's noise during the application of RF pulses harbours the risk that some participants will feel uncomfortable. To obtain noise suppression participants will wear earplugs. During the entering of the MRI tube it can happen that some participants feel some sort of dizziness as their vestibular organs have to adapt to the stronger magentic field of 3T. For the unprobable case that they don't feel well during the scans they will have the possibility at every moment to press an emergency switch to stop the scanning process. All the MR scans will be supervised by experienced medical technologists. 

Patients receive information about the procedure for incidental findings. If neurological abnormalities are detected, patients are informed and given instructions about the necessary steps that should be taken, e.g. a neurological examination.

The TMS sessions will be supervised by a medical doctor. TMS does not harbour any major health risk. The application of the electric current through the coil and thus the entering of the magnetic fields into the scalp can sometimes cause an itching pain on the cranial skin. Also sometimes participants report nausea or involontary movements across the face. In rare instances participants can have syncopes. A syncope does not represent a lief-threatening condition. To avoid the risk of an epileptic seizure participants with higher risk because of any threshhold lowering medication or history of epileptic seizures in the familiy will be excluded. Also after the sessions, participants are given the chance to contat a medical doctor in case they feel not well after stimulation. 

In case of a strong craving after the presentation of the alcoholic stimuli, participants can seek support from psychologists in the substance abuse clinics.
 

\subsection{Timetable and milestones}


\subsection{Significance of planned research to the scientific community and to potential users}


\printbibliography  
\end{document}